\documentclass[12pt,a4paper]{article}
\usepackage[margin=0.9in]{geometry}
\usepackage{fancyhdr}
\usepackage{graphicx}
\usepackage{listings}
\usepackage{xcolor}
\usepackage{hyperref}
\usepackage{longtable}
\usepackage{booktabs}
\usepackage{caption}

% Configure listings for code display
\definecolor{codegreen}{rgb}{0,0.6,0}
\definecolor{codegray}{rgb}{0.5,0.5,0.5}
\definecolor{codepurple}{rgb}{0.58,0,0.82}
\definecolor{backcolour}{rgb}{0.95,0.95,0.92}

\lstdefinestyle{mystyle}{
    backgroundcolor=\color{backcolour},
    commentstyle=\color{codegreen},
    keywordstyle=\color{codepurple},
    numberstyle=\tiny\color{codegray},
    stringstyle=\color{codepurple},
    basicstyle=\ttfamily\footnotesize,
    breakatwhitespace=false,
    breaklines=true,
    captionpos=b,
    keepspaces=true,
    numbers=left,
    numbersep=5pt,
    showspaces=false,
    showstringspaces=false,
    showtabs=false,
    tabsize=2
}

\lstset{style=mystyle}

% Configure header and footer
\pagestyle{fancy}
\fancyhf{}
\rhead{SPE Mini Project}
\lhead{Scientific Calculator}
\cfoot{\thepage}

\title{\huge\textbf{Scientific Calculator} \\ \Large\textit{Mini Project Report}}
\author{Rahul Raman}
\date{\today}

\begin{document}

\maketitle

\newpage

\section{Project Overview}

\subsection{Introduction}
This report documents the development and deployment pipeline of a \textbf{Scientific Calculator} application built in Java. The project demonstrates a complete software development lifecycle including continuous integration, automated testing, containerization, and infrastructure-as-code deployment strategies.

\textbf{Repository:} \url{https://github.com/rahul09123/SPE-Mini-Project}

\begin{center}
    \includegraphics[width=0.5\textwidth]{Images/Calculator.png}
    \captionof{figure}{Scientific Calculator Application}
\end{center}

\subsection{Tech Stack \& Objectives}
\textbf{Stack:} Java 17, Maven 3, JUnit 5, Jenkins, Docker, Ansible

\textbf{Application:} Command-line scientific calculator with basic arithmetic (add, subtract, multiply, divide) and advanced functions (square root, factorial, logarithm, power) with comprehensive error handling.

\section{Jenkins Pipeline Architecture}

\subsection{Pipeline Overview}
The Jenkins pipeline automates the complete build, test, and deployment workflow. It follows a declarative pipeline approach, providing clarity and maintainability.

\begin{center}
    \includegraphics[width=0.65\textwidth]{Images/JenkinsPipeline.png}
    \captionof{figure}{Jenkins Pipeline Architecture}
\end{center}

\subsection{Pipeline Stages}

\subsubsection{Stage 1: Clone Git}
\begin{lstlisting}[language=groovy,caption=Clone Git Stage]
stage('Clone Git') {
    steps {
        script {
            git branch: 'main',
                credentialsId: 'github_credentials',
                url: "${GITHUB_REPO_URL}"
        }
    }
}
\end{lstlisting}

\textbf{Purpose:} Fetches latest source code from GitHub. Authenticates to repository and provides clean workspace for builds.

\subsubsection{Stage 2: Build}
\begin{lstlisting}[language=groovy,caption=Build Stage]
stage('Build') {
    steps {
        sh 'mvn clean package -DskipTests'
    }
}
\end{lstlisting}

\textbf{Purpose:} Compiles Java source and creates JAR artifact. Tests are skipped here and executed in the Test stage.

\subsubsection{Stage 3: Test}
\begin{lstlisting}[language=groovy,caption=Test Stage]
stage('Test') {
    steps {
        sh 'mvn test'
    }
}
\end{lstlisting}

\textbf{Purpose:} Executes 47 comprehensive unit tests using JUnit 5 and generates test reports.

\subsubsection{Stage 4: Verify JAR Existence}
\begin{lstlisting}[language=groovy,caption=Verify JAR Existence Stage]
stage('Verify JAR Existence') {
    steps {
        sh 'ls -lh target/'
    }
}
\end{lstlisting}

\textbf{Purpose:} Verifies successful JAR creation by listing target directory contents.

\subsubsection{Stage 5: Build and Push Docker Image}
\begin{lstlisting}[language=groovy,caption=Docker Build Stage (Partial)]
stage('Build and Push Docker Image') {
    steps {
        script {
            sh """
            docker buildx create --use || true
            docker buildx inspect --bootstrap

            docker buildx build \
              --platform linux/amd64,linux/arm64 \
              -t ${DOCKER_HUB_USERNAME}/${DOCKER_IMAGE_NAME}:latest \
              --push .
            """
        }
    }
}
\end{lstlisting}

\textbf{Purpose:} Builds and pushes multi-platform Docker images (linux/amd64, linux/arm64) to Docker Hub registry.

\begin{center}
    \includegraphics[width=0.5\textwidth]{Images/DockerHub.png}
    \captionof{figure}{Docker Hub Multi-Platform Build}
\end{center}

\subsubsection{Stage 6: Deploy with Ansible}
\begin{lstlisting}[language=groovy,caption=Ansible Deployment Stage]
stage('Deploy with Ansible') {
    steps {
        sh 'ansible-playbook -i inventory.ini deploy.yml'
    }
}
\end{lstlisting}

\textbf{Purpose:} Automates application deployment to target servers using Ansible playbooks.

\begin{center}
    \includegraphics[width=0.5\textwidth]{Images/container\ Running.png}
    \captionof{figure}{Container Running and Deployed}
\end{center}

\subsection{Environment Variables}

\begin{lstlisting}[language=groovy,caption=Environment Configuration]
environment {
    DOCKER_IMAGE_NAME = 'scientific-calculator'
    GITHUB_REPO_URL = 'https://github.com/rahul09123/SPE-Mini-Project.git'
    DOCKER_HUB_USERNAME = 'rahul0129'
    DOCKER_HOST = "unix:///Users/rahulraman/.docker/run/docker.sock"
}
\end{lstlisting}

These variables configure: Docker image name, GitHub repository URL, Docker Hub credentials, and Docker daemon socket.


\section{Testing and Validation}

\subsection{Test Suite Overview}
The project includes 47 comprehensive unit tests covering all calculator operations. Tests are organized using JUnit 5's annotation-based approach.

\subsection{Test Coverage}

\subsubsection{Mathematical Operations Tested}
\begin{itemize}
    \item \textbf{Square Root:} Positive numbers, zero, decimals, negative numbers
    \item \textbf{Factorial:} Zero, one, positive integers (up to 20)
    \item \textbf{Natural Logarithm:} One, Euler's number, positive values, zero, negative values
    \item \textbf{Power Operations:} Base cases, decimal exponents, edge cases
\end{itemize}

\subsubsection{Test Sample}
Tests validate: Square Root ($\sqrt{16} = 4$, $\sqrt{0} = 0$), Factorial ($0! = 1$, $5! = 120$, $20! = 2.4{\times}10^{18}$), Natural Logarithm ($\ln(1) = 0$, $\ln(e) = 1$). Edge cases include NaN handling for negative numbers and infinity for $\ln(0)$. All 47 tests execute in 41 milliseconds with 100\% pass rate.


\section{Deployment Configuration}

\subsection{Ansible Deployment Playbook (deploy.yml)}
The \texttt{deploy.yml} file automates container deployment to target servers using Ansible orchestration.

\subsubsection{Playbook Structure}
\begin{lstlisting}[language=yaml,caption=deploy.yml Tasks]
- name: Pull the latest Docker image
  shell: docker pull rahul0129/scientific-calculator:latest

- name: Stop and remove existing container
  shell: docker rm -f calculator
  ignore_errors: true

- name: Run calculator container (detached)
  shell: docker run -dit --name calculator \
    --restart unless-stopped \
    rahul0129/scientific-calculator:latest

- name: Display container status
  shell: docker ps -f "name=calculator"
\end{lstlisting}

\subsubsection{Key Operations}
\begin{itemize}
    \item \textbf{Pull Image:} Downloads latest Docker image from Docker Hub
    \item \textbf{Cleanup:} Removes existing container (non-blocking with \texttt{ignore\_errors})
    \item \textbf{Deploy:} Launches container in detached mode with auto-restart policy
    \item \textbf{Verify:} Confirms successful container deployment and status
\end{itemize}

This playbook targets \texttt{webservers} host group defined in \texttt{inventory.ini}, enabling centralized deployment across multiple servers.

\subsection{Pipeline Execution Flow}
The Jenkins pipeline executes sequentially through all 6 stages: Clone (fetch code) → Build (Maven compile) → Test (JUnit 5, 47 tests) → Verify (JAR check) → Containerize (Docker multi-platform) → Deploy (Ansible).

\subsection{Quality Assurance \& Benefits}
\textbf{QA Measures:} Automated testing with 47 test cases, build verification, multi-platform testing, and Infrastructure-as-Code deployment ensure reliability.

\textbf{Benefits:} Complete automation, consistency across environments, scalability to multiple servers, high portability across CPU architectures, and rapid feedback (41ms test execution time).

\newpage

\section{Conclusion}

This project successfully demonstrates a complete CI/CD pipeline implementation with 100\% test success (47/47 passing), fast execution (41ms), and multi-platform Docker containerization. The automated pipeline integrates Git-based version control, Maven-based Java builds, comprehensive JUnit 5 testing, multi-platform Docker image building and pushing, and Ansible-based infrastructure deployment. This comprehensive approach showcases modern DevOps practices combining continuous integration, automated testing, and infrastructure-as-code principles.\vspace{0.3cm}

\begin{center}
    \includegraphics[width=0.5\textwidth]{Images/webHook.png}
    \captionof{figure}{GitHub WebHook Configuration}
\end{center}

\vspace{0.2cm}

\begin{center}
    \includegraphics[width=0.5\textwidth]{Images/emailNotification.png}
    \captionof{figure}{Email Notification on Build Status}
\end{center}

\end{document}
